\documentclass{article}

\usepackage{url} 

\usepackage{pdfpages}
\usepackage{lastpage}
\usepackage{fancyhdr}
\usepackage{ngerman}
\usepackage{listings}

\usepackage{tabularx}
\usepackage{floatrow}
\usepackage[tableposition=top]{caption}
\floatsetup[table]{capposition=top}

\usepackage{amsmath, amssymb}

\usepackage[utf8]{inputenc}


\usepackage[numbib]{tocbibind}



\newcommand\twodigits[1]{%
   \ifnum#1<10 0#1\else #1\fi
}



\lhead{Interferometer}
\rhead{13. November 2020\\T. Maier, J. Winkler}
%\cfoot{\twodigits{\thepage}~/ \pageref{LastPage}}
\cfoot{{\thepage}~/ \pageref{LastPage}}

\newcommand{\W}{\text{W}}
\newcommand{\V}{\text{V}}
\newcommand{\A}{\text{A}}


\newcommand{\mini}{\operatorname{min}}


\begin{document}

\parindent0cm

\includepdf{Deckblatt.pdf}


\pagestyle{fancy}

\section{Aufgabenstellung}

\begin{enumerate}
\item Demonstration und Erklärung des Einflusses der Größe einer Lichtquelle auf das Interferenzmuster eines Doppelspaltes.
\item Demonstration und Erklärung des Einflusses der spektralen Breite des Lichtes einer räumlich kohärenten Lichtquelle auf das Interferenzmuster eines Doppelspaltes.
\item Bestimmung der Dicke einer Kunststoffschicht mit dem Doppelspalt Interferenzmuster.
\item Bestimmung der Größe der Lichtquelle, bei der für Doppelspalten mit unterschiedlichem Spaltabstand das Licht noch räumlich kohärent ist.
\end{enumerate}



\section{Voraussetzungen und Grundlagen}

%\begin{figure}[H]
%\caption{Transformator}
%\label{fig:transformator}
%{\centering
%\includegraphics[scale=0.4]{transformator.png}
%~
%}
%\end{figure}





\section{Geräteliste}

\begin{table}[H]
\caption{Liste der verwendeten Geräte}

~

\begin{tabular}{l|p{3cm}p{3cm}llll}
Abk. & Bezeichnung  & Typ & Gerätenummer & Unsicherheit \\
\hline
S & Spaltblende \\
\hline
L1 & Sammellinse 1 & $f_1 = 300~$mm & & $\Delta f_1 = 0.5~$mm \\
\hline
L2 & Sammellinse 2 & $f_2 = 150~$mm & & $\Delta f_2 = 0.5~$mm \\
\hline
L3 & Sammellinse 3 & $f_3 = 40~$mm & & $\Delta f_3 = 0.5~$mm \\
\hline
L4 & Sammellinse 4 & $f_4 = 30~$mm & & $\Delta f_4 = 0.5~$mm \\
\hline
DS & Doppelspalt \\
\hline





B & Lochblenden und Irisblende & $d_1=2~$mm ~ ~ ~ ~ $d_2=3~$mm   ~~~~~~~~~~~~ $d_3=6~$mm & &  $\Delta d = 0.1~$mm \\
\hline
F & Filterrad für LEDs & \\
\hline
K & Kamera
\end{tabular}

\end{table}



\section{Beschreibung der Versuchsanordnung}








\section{Versuchsdurchführung und Messwerte}



\section{Auswertung}








\section{Zusammenfassung und Diskussion}




%\newpage 
%\appendix
%\section{Python Skript}



\definecolor{commentgreen}{RGB}{2,112,10}
\definecolor{eminence}{RGB}{108,48,130}
\definecolor{weborange}{RGB}{255,165,0}
\definecolor{frenchplum}{RGB}{129,20,83}

\lstdefinelanguage{python}{
    morekeywords={def, for, range, abs, return},
    otherkeywords={<-,->, |>, \%\{, \}, \{, \, (, )},
    sensitive=true,
    morecomment=[l]{\#},
    morecomment=[n]{/*}{*/},
    morecomment=[s][\color{purple}]{:}{\ },
    morestring=[s][\color{orange}]"",
    commentstyle=\color{commentgreen},
    keywordstyle=\color{eminence},
    stringstyle=\color{red},
	basicstyle=\ttfamily,
	breaklines,
	showstringspaces=false,
	frame=tb
}
%\lstinputlisting[language=Python,captionpos=b, label=lst:test,caption={Python Skript}]{generate_numbers.py}

%\lstinputlisting[language=Python,captionpos=b, label=lst:test,caption={Bessel Auswertung}]{generate_numbers_bessel.py}


%\lstinputlisting[language=Python,captionpos=b, label=lst:test,caption={Zerstreuungslinse Auswertung}]{generate_numbers_zerstreuungslinse.py}


\begin{thebibliography}{9}
\bibitem{quelle1} \url{https://www.youtube.com/watch?v=oFJCEGcwUiQ}, 07.11.2020, 00:15 Uhr
\bibitem{quelle2} \url{https://www.spektrum.de/lexikon/physik/abbesche-theorie/13}, 07.11.2020, 00:17 Uhr
\bibitem{quelle3} \url{https://www.univie.ac.at/mikroskopie/1_grundlagen/optik/opt_instrumente/7_abbe.htm}, 07.11.2020, 00:24 Uhr
\bibitem{quelle4} \url{https://physik.cosmos-indirekt.de/Physik-Schule/Rayleigh-Kriterium}, 07.11.2020, 00:26 Uhr
\bibitem{quelle5} \url{https://www.youtube.com/watch?v=PZaUY45ce8k}, 07.11.2020, 00:27 Uhr
\bibitem{quelle6} Unterlagen aus Moodle, H. Ditlbacher, bereitgestellt von der KF Universität Graz
\end{thebibliography}


\end{document}
