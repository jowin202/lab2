\documentclass{article}

\usepackage{url} 

\usepackage{pdfpages}
\usepackage{lastpage}
\usepackage{fancyhdr}
\usepackage{ngerman}
\usepackage{listings}

\usepackage{floatrow}
\usepackage[tableposition=top]{caption}
\floatsetup[table]{capposition=top}

\usepackage{amsmath, amssymb}

\usepackage[utf8]{inputenc}


\usepackage[numbib]{tocbibind}



\newcommand\twodigits[1]{%
   \ifnum#1<10 0#1\else #1\fi
}



\lhead{Transformator}
\rhead{23. Oktober 2020\\T. Maier, J. Winkler}
%\cfoot{\twodigits{\thepage}~/ \pageref{LastPage}}
\cfoot{{\thepage}~/ \pageref{LastPage}}

\newcommand{\W}{\text{W}}
\newcommand{\V}{\text{V}}
\newcommand{\A}{\text{A}}





\begin{document}

\parindent0cm

\includepdf{Deckblatt.pdf}


\pagestyle{fancy}

\section{Aufgabenstellung}






\section{Grundlagen und Versuchsaufbau}




\begin{figure}[H]
\caption{Versuchsaufbau Transformator. Tr1 Regeltrenntrafo, Tr2 Messtrafo, $R_s$ Shunt ($0.5~\Omega$), $I_1$ Primärstrom, $I_2$ Sekundärstrom, $U_1$ Primärspannung, $U_2$ Sekundärspannung, $N_{1W}$ Leistungsmessung}
\label{fig:pic1}
{\centering
\includegraphics[scale=0.9]{pic1.png}
~
}
\end{figure}



\section{Geräteliste}

\begin{table}[H]
\caption{Liste der verwendeten Geräte}

~

\begin{tabular}{l|llll}
Kürzel & Bezeichnung & Hersteller & Gerätenummer & Unsicherheit \\
\hline
DM & Digitalmultimeter & Leybold \\
TF & Transformator & Ruhstrat \\
A1 & Amperemeter 1 & Norma & & $\pm 1.5\%$ \\
A2 & Amperemeter 2 & Norma & & $\pm 1.5\%$ \\
V1 & Voltmeter 1 & Norma & & $\pm 0.5\%$ \\
V2 & Voltmeter 2 & Norma & VII/1121/3 & $\pm 0.5\%$ \\
SP & Spule \\
WS & Widerstand  & & VII/695 \\
LWS & Lastwiderstand \\
OS & Oszilloskop & Rigol \\
TT & Trenntrafo & Ruhstrat 
\end{tabular}

\end{table}




\section{Durchführung und Messwerte}

\subsection{Leerlauf}

Zuerst wurde die Schaltung gemäß Abbildung \ref{fig:pic1} aufgebaut, jedoch ohne das Amperemeter für den Sekundärstrom. Dan wurde eine Primärspannung von $U_1= 160~$V angelegt. Die Unsicherheit bei Voltmetern sind $0.5\%$, bei Amperemetern $1.5\%$. Daher gilt insgesamt 
\begin{align*}
U_1 &= (160 \pm 1.2)~\V \\
U_2 &= (17.6 \pm 0.12)~\V \\
I_1 &= (0.2 \pm 0.009)~\A \\
P_1 &= (6.9 \pm 0.1)~\W
\end{align*}
Da der Transformator im Leerlauf war, ist $I_2 = 0$ zu setzen.


\subsection{Ohm'sche Last}

Hier wird derselbe Aufbau verwendet, jedoch zusätzlich mit einem Verbraucher an der Sekundärseite. Es wird hier zusätzlich zur Sekundärspannung auch der Sekundärstrom $I_2$ gemessen. Der variable Widerstand wurde so gewählt, dass $I_2 < 1~\A$ ist. Es gilt
\begin{align*}
U_1 &= (160 \pm 1.2)~\V \\
U_2 &= (16.6 \pm 0.12)~\V \\
I_1 &= (0.24 \pm 0.009)~\A \\
I_2 &= (0.68 \pm 0.02)~\A \\
P_1 &= (19.3\pm0.1)~\W
\end{align*}


\section{Auswertung}

\subsection{Leerlauf}
Für die Scheinleistung auf der Primärseite ergibt sich
\begin{align*}
S_1 = U_1\cdot I_1 = 32~\text{W}
\end{align*}
Die Fehlerrechnung ergibt
\begin{align*}
\Delta S_1 &= \Delta U_1\cdot I_1 + U_1\cdot \Delta I_1 = 1.68~\W \approx 2~\W
\end{align*}

Die Blindleistung ist
\begin{align*}
Q_1 = \sqrt{S_1^2 - P_1^2} = 31.25~\W
\end{align*}
Für die Fehlerrechnung gilt
\begin{align*}
\Delta Q_1 = \frac{S_1\cdot \Delta S_1}{\sqrt{S_1^2-P_1^2}} + \frac{P_1\cdot \Delta P_1}{\sqrt{S_1^2-P_1^2}} \approx 1.75~\W \approx 2~\W
\end{align*}

Der Leistungsfaktor ist
\begin{align*}
\cos(\phi) = \frac{P_1}{S_1} = 0.22
\end{align*}
Für die Fehlerrechnung gilt
\begin{align*}
\Delta\cos(\phi) = \frac{\Delta P_1}{S_1} + \frac{P_1}{S_1^2}\cdot \Delta S_1 = 0.01
\end{align*}

\subsection{Ohm'sche Last}

Analog zum Leerlauf gilt hier für die Scheinleistung
\begin{align*}
S_1 &= 37.6~\W \\
\Delta S_1 &= 1.7~\W
\end{align*}
Die Blindleistung ergibt
\begin{align*}
Q_1 &= 37~\W \\
\Delta Q_1 &= 1.8~\W
\end{align*}
Der Leistungsfaktor ist
\begin{align*}
\cos(\phi) &= 0.18 \\
\Delta \cos(\phi) &= 0.01
\end{align*}

Zusätzlich kann man jetzt die Sekundärseitige Wirkleistung berechnen (unter Annahme der Ohm'schen Last)
\begin{align*}
P_2 = U_2\cdot I_2 = 11.3~\text{W}
\end{align*}
Die Fehlerrechnung ergibt
\begin{align*}
\Delta P_2 &= \Delta U_2\cdot I_2 + U_2\cdot \Delta I_2 = 0.4~\W
\end{align*}


Der Wirkungsgrad kann folgend berechnet werden
\begin{align*}
\eta &= \frac{P_2}{P_1} = 0.58 \\
\Delta \eta &= \frac{\Delta P_2}{P_1} + \frac{P_2}{P_1^2}\cdot \Delta P_2 = 0.02
\end{align*}

Es fehlt noch die Verlustleistung und die dazugehörige Fehlerrechnung
\begin{align*}
P_V &= P_1-P_2 = 8.0~\W\\
\Delta P_V &= \Delta P_1 + \Delta P_2 = 0.5~\W
\end{align*}

\section{Zusammenfassung}

Für den Leerlauf gilt
\begin{align*}
S_1 &= (32 \pm 2)~\W \\
Q_1 &= (31 \pm 2)~\W \\
\cos(\phi) &= (0.22 \pm 0.01)
\end{align*}
Da $I_2=0$ ist, gilt natürlich auch $P_2=0$ und $\eta = 0$.


Für die Ohm'sche Last gilt
\begin{align*}
S_1 &= (38 \pm 2)~\W \\
Q_1 &= (37 \pm 2)~\W \\
\cos(\phi) &= (0.18 \pm 0.01) \\
P_2 &= (11.3 \pm 0.4)~\W \\
\eta &= (0.58 \pm 0.02) \\
P_V &= (8.0 \pm 0.5)~\W
\end{align*}


\section{Diskussion}



\begin{figure}[H]
\caption{Transformator im Leerlauf. Channel 1 ist proportional zum Primärstrom, Channel 2 ist Sekundärspannung}
\label{fig:task1}
{\centering
\includegraphics[scale=0.4]{task1.jpg}
~
}
\end{figure}

\begin{figure}[H]
\caption{Transformator mit Leerlauf. Channel 1 ist proportional zum Primärstrom, Channel 2 ist Sekundärspannung}
\label{fig:task1}
{\centering
\includegraphics[scale=0.4]{task1.jpg}
~
}
\end{figure}






%\newpage 
\appendix
\section{Python Skript}



\definecolor{commentgreen}{RGB}{2,112,10}
\definecolor{eminence}{RGB}{108,48,130}
\definecolor{weborange}{RGB}{255,165,0}
\definecolor{frenchplum}{RGB}{129,20,83}

\lstdefinelanguage{python}{
    morekeywords={def, for, range, abs, return},
    otherkeywords={<-,->, |>, \%\{, \}, \{, \, (, )},
    sensitive=true,
    morecomment=[l]{\#},
    morecomment=[n]{/*}{*/},
    morecomment=[s][\color{purple}]{:}{\ },
    morestring=[s][\color{orange}]"",
    commentstyle=\color{commentgreen},
    keywordstyle=\color{eminence},
    stringstyle=\color{red},
	basicstyle=\ttfamily,
	breaklines,
	showstringspaces=false,
	frame=tb
}
\lstinputlisting[language=Python,captionpos=b, label=lst:test,caption={Python Skript}]{generate_numbers.py}

%\lstinputlisting[language=Python,captionpos=b, label=lst:test,caption={Bessel Auswertung}]{generate_numbers_bessel.py}


%\lstinputlisting[language=Python,captionpos=b, label=lst:test,caption={Zerstreuungslinse Auswertung}]{generate_numbers_zerstreuungslinse.py}


\begin{thebibliography}{9}
\bibitem{chemie} \url{https://www.chemie.de/lexikon/Elektrochemisches_äquivalent.html}, 22.10.2020 22:53 Uhr
\bibitem{tu} bereitgestellte Unterlagen zum Versuch aus dem TeachCenter der TU Graz
\end{thebibliography}


\end{document}
