\documentclass{article}

\usepackage{url} 

\usepackage{pdfpages}
\usepackage{lastpage}
\usepackage{fancyhdr}
\usepackage{ngerman}
\usepackage{listings}

\usepackage{tabularx}
\usepackage{floatrow}
\usepackage[tableposition=top]{caption}
\floatsetup[table]{capposition=top}

\usepackage{amsmath, amssymb}

\usepackage[utf8]{inputenc}


\usepackage[numbib]{tocbibind}



\newcommand\twodigits[1]{%
   \ifnum#1<10 0#1\else #1\fi
}



\lhead{Abbe-Theorie}
\rhead{13. November 2020\\T. Maier, J. Winkler}
%\cfoot{\twodigits{\thepage}~/ \pageref{LastPage}}
\cfoot{{\thepage}~/ \pageref{LastPage}}

\newcommand{\W}{\text{W}}
\newcommand{\V}{\text{V}}
\newcommand{\A}{\text{A}}





\begin{document}

\parindent0cm

\includepdf{Deckblatt.pdf}


\pagestyle{fancy}

\section{Aufgabenstellung}

\begin{enumerate}
\item Qualitative  Untersuchung des Zusammenhangs zwischen der  Auflösung  des  Bildes  eines Spaltgitters (Testobjekt) und der Anzahl der transmittierten Beugungsordnungen.
\item Quantitative Bestimmung  des Auflösungsvermögens  einer  Linse  in Abhängigkeit  von  ihrer numerischen Apertur ($N_A$) für zwei unterschiedliche Wellenlängen der Beleuchtung.
\end{enumerate}


\section{Voraussetzungen und Grundlagen}

Die Abbesche Theorie besagt, dass das Auflösungsvermögen eines Objekts maßgeblich durch die Wellenlänge des eingestrahlten Lichts begrenzt wird. Das Auflösungsvermögen ist dabei die Fähigkeit eines Instruments Objektdetails noch getrennt abbilden zu können. Eine der Grundlage für diese Theorie bildet das Huygenssche Prinzip, wonach jeder Punkt einer Wellenfront der Ausgangspunkt einer neuen kugelförmigen Elementarwelle ist.

Licht trifft also auf ein Gitter, es bilden sich kugelförmige Wellen und es kann sowohl konstruktive als auch destruktive Interferenz beobachtet werden. Die Abbesche Theorie sagt hier, dass jedoch mindestens die Maxima nullter und erster Ordnung von einer Linse erfasst werden müssen, um überhaupt eine Auflösung zu erhalten. Außerdem kann die Linse (rein technisch) nicht unendlich groß gebaut (d.h. es können nie alle Maxima erfasst werden) und auch nicht beliebig nahe an das Gitter herangebracht werden. Der Zusammenhang zwischen Einfallswinkel $\alpha$ des Lichts, das höchstens von der Linse erfasst werden kann und Brechzahl n des Mediums, in dem sich die Apertur befindet, ist durch die sogenannte numerische Apertur $N_A$ (Auflösungvermögen des Mikroskops) gegeben
\begin{align}
N_A = n\cdot\sin(\alpha)
\end{align}
Betrachtet man nun zwei Punkte eines Gitters, die im Abstand $x$ voneinander entfernt sind und ebenfalls jeweils eine kugelförmige Welle aussenden, so ergibt sich der Zusammenhang
\begin{align}
\sin(\alpha) = \frac{\lambda}{x\cdot n}
\end{align}
bzw. durch Umformung
\begin{align}
\sin(\alpha) = \frac{\lambda}{n\cdot \sin(\alpha)} = \frac{\lambda}{N_A} 
\end{align}
wobei $\alpha$ hier dem Winkel zwischen nullter und erster Ordnung der Maxima und $\lambda$ der Wellenlänge des eingestrahlten Lichts entspricht. \cite{quelle1} \cite{quelle2} \cite{quelle3} 

Zum ähnlichen Überlegungen kommt man aufgrund des Rayleigh-Kriteriums zur optischen Mikroskopie, welches besagt, dass zwei Punkte im Abstand $x$ gerade dann noch auflösbar sind, wenn das Beugungsscheibchen des ersten Objekts auf das erste Minimum des Beugungsscheibchens des zweiten Objekts fällt \cite{quelle4} \cite{quelle5}
\begin{align}
x = \frac{\frac{1.22}{2}\cdot \lambda}{N_A}
\end{align}

%\begin{figure}[H]
%\caption{Transformator}
%\label{fig:transformator}
%{\centering
%\includegraphics[scale=0.4]{transformator.png}
%~
%}
%\end{figure}





\section{Geräteliste}

\begin{table}[H]
\caption{Liste der verwendeten Geräte}

~

\begin{tabular}{l|p{3cm}p{3cm}llll}
Abk. & Bezeichnung  & Typ & Gerätenummer & Unsicherheit \\
\hline
LA & Diodenge\-pumpter Festkörperlaser THOR-Labs  & Festkörperlaser $\lambda~=~531.9~$nm & LDS5 & $\Delta \lambda = 0.05~$nm \\
\hline
RL & Rote LED &  $\lambda=470~$nm & & $\Delta \lambda = 0.05~$nm \\
\hline
BL & Blaue LED &  $\lambda=635~$nm & & $\Delta \lambda = 0.05~$nm \\
\hline
TO & Testobjekt & Siehe Abb. \\
\hline
L1 & Abbildungslinse 1 & $f_1 = 200~$mm & & $\Delta f_1 = 1~$mm \\
\hline
L2 & Abbildungslinse 2 & $f_2= 60~$mm, freier Durchmesser: $21.4$ mm, achromatisches Linsenpaar & & $\Delta f_2 = 1~$mm\\
\hline
L3 & Hilfslinse & $f_3 = 50~$mm, klappbar & & $\Delta f_3 = 1~$mm \\
\hline
B & Lochblenden und Irisblende & $d_1=2~$mm ~ ~ ~ ~ $d_2=3~$mm   ~~~~~~~~~~~~ $d_3=6~$mm & &  $\Delta d = 0.1~$mm \\
\hline
F & Filterrad für LEDs & \\
\hline
K & Kamera
\end{tabular}

\end{table}



\section{Beschreibung der Versuchsanordnung}

\begin{figure}[H]
\caption{Das verwendete Testobjekt, USAF 1951. aus \cite{quelle6}}
\label{fig:usaf}
{\centering
\includegraphics[scale=1.5]{usaf.png}
~
}
\end{figure}


\begin{figure}[H]
\caption{Aufbau (inkl. Abmessungen) des Versuchs; $L_1$: $f_1= 200~$mm, $F$: Filterrad mit 2 LEDs, Graufilter und freiem Durchgang, $T$: Testobjekt; $L_2$: $f_2= 60~$mm; $B$: Filterrad mit 3 Lochblenden, einer Irisblende und einer Drahtblende, $L_3$ (einklappbar): $f_3= 50~$mm. Quelle: \cite{quelle6}}
\label{fig:usaf}
{\centering
\includegraphics[scale=1]{versuch.png}
~
}
\end{figure}



\begin{table}
\caption{Auflösungsvermögen je nach nach Element und Gruppe mit der Einheit mm$^{-1}$.}
\label{tab:aufl}
\begin{tabular}{|l||l|l|l|l|l|l|l|l|l|l|l|}
\hline
\textbf{Elem. Nr.} & \multicolumn{10}{|c|}{\textbf{Gruppen Nr.}}\\
\hline
& -2 & -1 & 0 & 1 & 2 & 3 & 4 & 5 & 6 & 7 \\
\hline
1 & 0.250 & 0.500 & 1.00 & 2.00 & 4.00 & 8.00 & 16.00 & 32.0 & 64.0 & 128.0 \\
\hline
2 & 0.280 & 0.561 & 1.12 & 2.24 & 4.49 & 8.98 & 17.95 & 36.0 & 71.8 & 144.0 \\
\hline
3 & 0.315 & 0.630 & 1.26 & 2.52 & 5.04 & 10.10 & 20.16 & 40.3 & 80.6 &  161.0 \\
\hline
4 & 0.353 & 0.707 & 1.41 & 2.83 & 5.66 & 11.30 & 22.62 & 45.3 & 90.5 & 181.0 \\
\hline
5 & 0.397 & 0.793 & 1.59 & 3.17 & 6.35 & 12.70 & 25.39 & 50.8 & 102.0 & 203.0 \\
\hline
6 & 0.445 & 0.891 & 1.78 & 3.56 & 7.13 & 14.30 & 28.50 & 57.0 & 114.0 & 228.0 \\
\hline
\end{tabular}
\end{table}







\section{Versuchsdurchführung und Messwerte}

\subsection{Beugungsordnungen und Auflösung}


Hierher kommen ganz viele Bilder und davor ein kurzer Text was wir tun

\newpage


\subsection{Auflösevermögen und Numerische Apertur}



Hierher kommen ganz viele Bilder und davor ein kurzer Text was wir tun

\newpage



\section{Auswertung}

\subsection{Beugungsordnungen und Auflösung}

Dieser Versuch läuft qualitativ ab. Hierbei zeigt sich in der Abbe-Theorie, dass in der 0. Beugungsordnung keine Information über die Struktur des Bildes übertragen wird. Um eine genauere Struktur zu erhalten, benötigt man also höhere Begungsordnungen.




\subsection{Auflösevermögen und Numerische Apertur}


Der Fehler wird nach der Größtfehlermethode berechnet und es gilt
\begin{align*}
x_b &= \frac{1}{a_{i,j}} \\
\Delta x_b &= \frac{1}{a_{i,j}^2}\cdot \Delta a_{i,j}
\end{align*}
wobei $a_{i,j}$ aus Tabelle~\ref{tab:aufl} genommen genommen wird. 


\begin{table}[H]
\caption{Gemessene Auflösungsvermögen für das blaue LED. $d$ Linsendurchmesser, $\Delta d = \pm 0.1$~mm, $E_b$ gefundenes Element für blaues Licht, $x_b$ Auflösungsvermögen vom blauen Licht (s. Tabelle \ref{tab:aufl}), $\Delta x_b$ Unsicherheit zum Auflösungsvermögen.}

\begin{tabular}{llll}
$d$ / mm & $E_b$ & $x_b$ / mm & $\Delta x_b$ / mm \\
\hline
2 & 6/1 & ... & .. \\
3 & 6/3 & ... & .. \\
6 & 7/1 & ... & .. 
\end{tabular}
\end{table}


\begin{table}[H]
\caption{Gemessene Auflösungsvermögen für das rote LED. $d$ Linsendurchmesser, $\Delta d = \pm 0.1$~mm, $E_r$ gefundenes Element für rote Licht, $x_r$ Auflösungsvermögen vom rotem Licht (s. Tabelle \ref{tab:aufl}), $\Delta x_b$ Unsicherheit zum Auflösungsvermögen.}

\begin{tabular}{llll}
$d$ / mm & $E_r$ & $x_r$ / mm & $\Delta x_r$ / mm \\
\hline
2 & 6/1 & ... & .. \\
3 & 6/3 & ... & .. \\
6 & 7/1 & ... & .. 
\end{tabular}
\end{table}






\section{Zusammenfassung und Diskussion}

Es zeigt sich hier deutlich, dass ein Zusammenhang zwischen der Schärfe des Bildes und der Anzahl der Beugungsordnungen gilt. Und zwar wird das Bild mit der Anzahl der verwendeten Beugungsordnungen schärfer.

~

Im zweiten Teil zeigt sich, dass das Bild bei rotem LED besser aufgelöst ist. Dies liegt an der höheren Wellenlänge vom roten Licht.

Zusätzlich muss auch der Durchmesser der Blende berücksichtigt werden. Bei höheren Durchmesser wird die Unsicherheit geringer. Das liegt daran, dass bei einem größeren Durchmesser mehrere Beugungsmaxima genutzt werden können.





%\newpage 
%\appendix
%\section{Python Skript}



\definecolor{commentgreen}{RGB}{2,112,10}
\definecolor{eminence}{RGB}{108,48,130}
\definecolor{weborange}{RGB}{255,165,0}
\definecolor{frenchplum}{RGB}{129,20,83}

\lstdefinelanguage{python}{
    morekeywords={def, for, range, abs, return},
    otherkeywords={<-,->, |>, \%\{, \}, \{, \, (, )},
    sensitive=true,
    morecomment=[l]{\#},
    morecomment=[n]{/*}{*/},
    morecomment=[s][\color{purple}]{:}{\ },
    morestring=[s][\color{orange}]"",
    commentstyle=\color{commentgreen},
    keywordstyle=\color{eminence},
    stringstyle=\color{red},
	basicstyle=\ttfamily,
	breaklines,
	showstringspaces=false,
	frame=tb
}
%\lstinputlisting[language=Python,captionpos=b, label=lst:test,caption={Python Skript}]{generate_numbers.py}

%\lstinputlisting[language=Python,captionpos=b, label=lst:test,caption={Bessel Auswertung}]{generate_numbers_bessel.py}


%\lstinputlisting[language=Python,captionpos=b, label=lst:test,caption={Zerstreuungslinse Auswertung}]{generate_numbers_zerstreuungslinse.py}


\begin{thebibliography}{9}
\bibitem{quelle1} \url{https://www.youtube.com/watch?v=oFJCEGcwUiQ}, 07.11.2020, 00:15 Uhr
\bibitem{quelle2} \url{https://www.spektrum.de/lexikon/physik/abbesche-theorie/13}, 07.11.2020, 00:17 Uhr
\bibitem{quelle3} \url{https://www.univie.ac.at/mikroskopie/1_grundlagen/optik/opt_instrumente/7_abbe.htm}, 07.11.2020, 00:24 Uhr
\bibitem{quelle4} \url{https://physik.cosmos-indirekt.de/Physik-Schule/Rayleigh-Kriterium}, 07.11.2020, 00:26 Uhr
\bibitem{quelle5} \url{https://www.youtube.com/watch?v=PZaUY45ce8k}, 07.11.2020, 00:27 Uhr
\bibitem{quelle6} Unterlagen aus Moodle, H. Ditlbacher, bereitgestellt von der KF Universität Graz
\end{thebibliography}


\end{document}
