\documentclass{article}

\usepackage{url} 

\usepackage{pdfpages}
\usepackage{lastpage}
\usepackage{fancyhdr}
\usepackage{ngerman}
\usepackage{listings}

\usepackage{floatrow}
\usepackage[tableposition=top]{caption}
\floatsetup[table]{capposition=top}

\usepackage{amsmath, amssymb}

\usepackage[utf8]{inputenc}


\usepackage[numbib]{tocbibind}



\newcommand\twodigits[1]{%
   \ifnum#1<10 0#1\else #1\fi
}



\lhead{Halbschattenpolarimeter}
\rhead{16. Oktober 2020\\T. Maier, J. Winkler}
%\cfoot{\twodigits{\thepage}~/ \pageref{LastPage}}
\cfoot{{\thepage}~/ \pageref{LastPage}}

\newcommand{\as}{\alpha_\text{spez}}

\begin{document}

\parindent0cm

\includepdf{Deckblatt.pdf}


\pagestyle{fancy}

\section{Aufgabenstellung}

\begin{enumerate}
\item Bestimmung der Konzentration einer Rohrzuckerlösung.
\item Bestimmung der spezifischen Drehung einer Quarzplatte.
\end{enumerate}

\section{Grundlagen und Versuchsaufbau}

Unter optischer Aktivität versteht man die Eigenschaft mancher Medien, die Schwingungsebene linear polarisierten Lichts nach rechts oder links zu drehen. Dieses Verhalten wird durch eine schraubenfömige Struktur des Kristallgitters (bei festen Stoffen) bzw. der Moleküle (bei Flüssigkeiten und Gasen) hervorgerufen. Denkt man sich die linear polarisierte Welle aus zweientgegengesetzt zirkular polarisierten Wellen zusammengesetzt, können die beiden Anteile in einem solchen Medium unterschiedliche Ausbreitungsgeschwindigkeiten besitzen. Sie treten daher aus dem Medium mit einer Phasendifferenz aus, und setzen sich zu einer linear polarisierten Welle mit gedrehter Polarisationsrichtung zusammen. Die Größe des Winkels $\alpha$ ist von der durchlaufenen Wegstrecke $d$ bzw. $\ell$, der spezifischen Drehung $\as$ und der Konzentration des Mediums $c$ abhängig.

\begin{equation}
\label{eq:main}
\begin{aligned}
  \alpha &= \as\cdot \ell \cdot c && \text{für Flüssigkeiten und Gase} \\
  \alpha &= \as\cdot d && \text{für Kristalle und Festkörper}
\end{aligned}
\end{equation}

Um aus natürlichem Licht linear polarisiertes zu erhalten, kann z.B. ein Nicol'sches Prismaverwendet werden. Es besteht aus einem geeignet ausgerichteten und geschnittenen doppelbrechenden Kristall. Dabei wird eintretendes unpolarisiertes Licht in zwei senkrecht zueinanderstehende linear polarisierte Anteile mit unterschiedlicher Ausbreitungsgeschwindigkeit aufgespalten. Durch Totalreflexion eines der beiden Anteile an einer Grenzfläche können die beiden Polarisationsanteile voneinander getrennt werden.

Da es aus physiologischen Gründen schwierig ist, eine Polarisator – Analysator – Anordnung aufmaximale oder minimale Helligkeit abzugleichen, wird für Messzwecke ein sog. Halbschattenpolarimeter (Abb. \ref{fig:halbschattenpolarimeter}) verwendet. Dabei wird nach dem Polarisator ein nur den halben Strahlengangbedeckendes zweites Nicol'sches Prisma eingesetzt, das um einen kleinen Winkel gegen den Pola-risator verdreht ist. Beim Durchstimmen des Analysators beobachtet man in dem nun geteiltenGesichtsfeld nacheinander ein Helligkeitsminimum in beiden Hälften. Befindet sich der Analysator genau senkrecht zum Symmetriewinkel zwischen Polarisator und Halbschattenprisma,erscheinen beide Gesichtsfeldhälften gleich hell. Die Trennlinie ist nicht sichtbar. Bei einer ge-ringen Abweichung von dieser Stellung wird eine der beiden Hälften sofort dunkler. Dadurch istein präziser Abgleich der Polarisator – Analysatorstellung möglich.


\begin{figure}[H]
\centering
\includegraphics[scale=0.4]{halbschattenpolarimeter.png}
\caption{Aufbau des Halbschattenpolarimeters $L$ Na-Dampflampe, $K$ Kondensor, $P_1$ Polarisator, $P_2$ Halbschattenprisma, $AS$ optisch aktive Substanz, $A$ Analysator, $F$ Fernrohr}
\label{fig:halbschattenpolarimeter}
\end{figure}

\newpage

\section{Geräteliste}

\begin{table}[H]
\caption{Liste der verwendeten Geräte}

~

\begin{tabular}{l|llll}
Bezeichnung & Gerätenummer & Unsicherheit \\
\hline
Micrometer & & $\pm~0.01~$mm \\
1. Halbschattenpolarimeter & & $\pm~0.05~^\circ$ \\
2. Halbschattenpolarimeter & & $\pm~0.1~^\circ$ \\
Schiebelehre & & $\pm~0.05~$mm \\
Rohrzuckerlösung & & \\
Quarzkristalle & & 
\end{tabular}

\end{table}




\section{Durchführung und Messwerte}

\subsection{Rohrzuckerlösung}

Zuerst wurde die Gesamtlänge des Probenglases mithilfe der Schiebelehre gemessen. Dann wurde die effektive Länge des Probenglases, also der Teil, der auch tatsächlich mit der Rohrzuckerlösung gefüllt war, sowohl innen als auch außen gemessen. Die Messvorgänge wurden jeweils fünf Mal wiederholt. Alle Werte sind in Tabelle \ref{tab:rohrzucker_abmessungen} zusammengefasst.

\begin{table}[H]
\caption{Rohrzuckerlösung in Probenglas, Abmessungen. $\ell_\text{ges}$ Gesamtlänge, $\ell_1$ innere Länge 1, $\ell_2$ innere Länge 2.}
\label{tab:rohrzucker_abmessungen}
\centering
\begin{tabular}{rrr}\\
 $\ell_\text{ges}$ / mm & $\ell_1$ / mm & $\ell_2$ /mm \\
 \hline
111.70 & 3.85 & 3.85\\
111.65 & 3.80 & 3.85\\
111.70 & 3.85 & 3.80\\
111.70 & 3.85 & 3.85\\
111.60 & 3.85 & 3.85
\end{tabular}
\end{table}


Um die Größe des Winkel $\alpha$ zu bestimmen, musste der Halbschattenpolarimeter zunächst auf einen offset Winkel hin kalibriert werden. Dazu wurde der Übergang von \textit{dunkel-hell-dunkel} zu \textit{hell-dunkel-hell} gesucht und der dabei entstehende Winkel (idealerweise 0°) gemessen. Danach konnte das Probenglas in den Strahlengang eingelegt werden und der Winkel beim Übergang wurde erneut gemessen. Die Messungen wurden jeweils zehn Mal durchgeführt und in Tabelle \ref{tab:rohrzucker_winkel} zusammengefasst.



\begin{table}[H]
\caption{Offset und Drehwinkel der Rohrzuckerlösung.}
\label{tab:rohrzucker_winkel}
\centering
\begin{tabular}{r|r}\\
 $\alpha_\text{off}$ / ${}^\circ$ & $\alpha_1$ / ${}^\circ$  \\
 \hline
0.00 & 4.40\\
0.00 & 4.40\\
0.00 & 4.50\\
0.00 & 4.60\\
0.00 & 4.70\\
0.00 & 4.40\\
0.00 & 4.40\\
0.00 & 4.45\\
0.00 & 4.50\\
0.00 & 4.35
\end{tabular}
\end{table}


\subsection{Quarzkristalle}

Als erstes wurden die vier Quarzkristalle mithilfe einer Micrometerschraube jeweils fünf Mal vermessen. Die Werte sind in Tabelle \ref{tab:kristalle_dicken} zusammengefasst.


\begin{table}[H]
\caption{Dicken der Quarzkristalle.}
\label{tab:kristalle_dicken}
\centering
\begin{tabular}{rrrr}\\
 $d_1$ / mm & $d_2$ / mm & $d_3$ / mm & $d_4$ / mm  \\
 \hline
6.98 & 6.51 & 9.17 & 9.43\\
6.99 & 6.50 & 9.16 & 9.43\\
6.99 & 6.50 & 9.17 & 9.42\\
6.99 & 6.51 & 9.16 & 9.43\\
6.99 & 6.51 & 9.16 & 9.42\\
6.98 & 6.50 & 9.16 & 9.43\\
6.99 & 6.51 & 9.17 & 9.42\\
6.98 & 6.50 & 9.16 & 9.44\\
6.99 & 6.50 & 9.16 & 9.43\\
6.99 & 6.51 & 9.16 & 9.43
\end{tabular}
\end{table}



Dann wurde auch hier wieder zunächst der Halbschattenpolarimeter mithilfe des \textit{dunkel-hell-dunkel} und \textit{hell-dunkel-hell} Übergangs auf einen offset Winkel kalibriert und dann die Proben nacheinander eingelegt und wieder dieser Übergang gesucht. Die Messungen wurden sowohl für den offset Winkel als auch für die unterschiedlichen Quarzkristalle fünf Mal durchgeführt und in Tabelle \ref{tab:kristall_winkel} notiert.


\begin{table}[H]
\caption{Offset und Drehwinkel der Quarzkristalle.}
\label{tab:kristall_winkel}
\centering
\begin{tabular}{r|rrrr}\\
 $\alpha_\text{off}$ / ${}^\circ$ & $\alpha_1$ / ${}^\circ$ & $\alpha_2$ / ${}^\circ$ & $\alpha_3$ / ${}^\circ$ & $\alpha_4$ / ${}^\circ$  \\
 \hline
0.00 & 28.85 & 39.75 & 160.65 & 25.70\\
0.00 & 28.60 & 39.80 & 160.60 & 25.35\\
0.00 & 28.40 & 39.00 & 160.25 & 25.30\\
0.00 & 28.70 & 39.60 & 160.40 & 26.00\\
0.00 & 28.00 & 39.80 & 160.60 & 25.50
\end{tabular}
\end{table}


\section{Auswertung}


\subsection{Rohrzuckerlösung}
Zuerst wird die Länge des Probenglases bestimmt werden, in dem sich die Flüssigkeit befindet.
\begin{align*}
\ell = \ell_\text{ges} - \ell_1 - \ell_2
\end{align*}
Mit Fehlerrechnung ergibt sich
\begin{align*}\\
\ell = (103.99 \pm0.03)~\text{mm} \\
\end{align*}\\


Bestimmung der Konzentration mit Gleichung \eqref{eq:main}
\begin{align*}
c &= \frac{\alpha}{\as\cdot \ell} \\
\Delta c &= \frac{\Delta \alpha}{\as\cdot \ell} +\frac{\alpha}{\as\cdot \ell^2}\cdot \Delta \ell
\end{align*}
mit $\as = 66.5~\frac{{}^\circ\cdot \text{cm}^3}{\text{dm}\cdot \text{g}}$. Es ergibt sich
\begin{align*}\\
c = (99.55 \pm0.75)~\text{mg}/\text{cm}^3 \\
\end{align*}\\



\subsection{Quarzkristalle}

Hier wird tabellarisch ausgewertet mit der Formel
\begin{align*}
\beta_i = \frac{\overline{\alpha_i-\alpha_{\text{off}}} + x}{\overline{d_i}}
\end{align*}
wobei die Notation mit dem obigen Querstrich für das arithmetische Mittel steht. $x$ ist ein Winkel aus der Menge $\{0, 180^\circ, -180^\circ, 360^\circ, -360^\circ\}$.


\begin{table}[H]
\caption{Auswertung der Drehwinkel.}
\label{tab:kristall_auswertung}
\centering
\begin{tabular}{rrrrr}\\
 $x$ / ${}^\circ$ & $\beta_1$ / ${}^\circ$ & $\beta_2$ / ${}^\circ$ & $\beta_3$ / ${}^\circ$ & $\beta_4$ / ${}^\circ$  \\
 \hline
$0~^\circ$ & 4.12 & 5.93 & -2.18 & 2.73\\
$180~^\circ$ & 29.88 & 33.60 & 17.46 & 21.82\\
$-180~^\circ$ & -21.65 & -21.73 & -21.82 & -16.37\\
$360~^\circ$ & 55.65 & 61.27 & 37.10 & 40.92\\
$-360~^\circ$ & -47.41 & -49.40 & -41.47 & -35.47
\end{tabular}
\end{table}



\section{Zusammenfassung und Diskussion}



%\newpage 
%\appendix
%\section{Python Skript}



\definecolor{commentgreen}{RGB}{2,112,10}
\definecolor{eminence}{RGB}{108,48,130}
\definecolor{weborange}{RGB}{255,165,0}
\definecolor{frenchplum}{RGB}{129,20,83}

\lstdefinelanguage{python}{
    morekeywords={def, for, range, abs, return},
    otherkeywords={<-,->, |>, \%\{, \}, \{, \, (, )},
    sensitive=true,
    morecomment=[l]{\#},
    morecomment=[n]{/*}{*/},
    morecomment=[s][\color{purple}]{:}{\ },
    morestring=[s][\color{orange}]"",
    commentstyle=\color{commentgreen},
    keywordstyle=\color{eminence},
    stringstyle=\color{red},
	basicstyle=\ttfamily,
	breaklines,
	showstringspaces=false,
	frame=tb
}
%\lstinputlisting[language=Python,captionpos=b, label=lst:test,caption={Laplace Auswertung}]{generate_numbers_laplace.py}

%\lstinputlisting[language=Python,captionpos=b, label=lst:test,caption={Bessel Auswertung}]{generate_numbers_bessel.py}


%\lstinputlisting[language=Python,captionpos=b, label=lst:test,caption={Zerstreuungslinse Auswertung}]{generate_numbers_zerstreuungslinse.py}




\end{document}
